%---------------------------------------------------------
% Define coordinates and shifts for placement
%---------------------------------------------------------
\def\xi{-2.3}   % left x-position for "Random noise / dataset"
\def\xii{4}     % right x-position for "Fake samples / Encoded dataset"
\def\xiv{6}     % far-right x-position for MMD block
\def\yi{0.5}    % top y reference
\def\ximg{1}    % x-position of central image (qag_small)
\def\yimg{0}    % y-position of central image (qag_small)
\def\YSHIFT{-1.6} % vertical offset for "True dataset" and "Encoded dataset"

% Vertical spacing for stacked nodes
\def\yishift{0.4}
\FPeval{\yii}{clip(\yi - \yishift)}
\FPeval{\yiii}{clip(\yii - \yishift)}

% Compute shifted y positions (lower block)
\FPeval{\Yi}{clip(\YSHIFT + \yi)}
\FPeval{\Yii}{clip(\YSHIFT + \yii)}
\FPeval{\Yiii}{clip(\YSHIFT + \yiii)}

%---------------------------------------------------------
% Arrow macros
%---------------------------------------------------------
\newcommand{\harr}[2]{%
% Short horizontal arrow: (#1,#2) → (#1+0.6,#2)
\FPeval{\newx}{clip(#1 + 0.6)}%
\draw[
-{Triangle[width=5pt,length=4pt]},
line width=2pt,
color=\tertiarycolor
] (#1, #2) -- (\newx,#2);
}

\newcommand{\longharr}[2]{%
% Long horizontal arrow: (#1,#2) → (#1+4.5,#2)
\FPeval{\newx}{clip(#1 + 4.5)}%
\draw[
-{Triangle[width=5pt,length=4pt]},
line width=2pt,
color=\tertiarycolor
] (#1, #2) -- (\newx,#2);
}

%---------------------------------------------------------
% Main TikZ picture
%---------------------------------------------------------
\begin{tikzpicture}

	%-------------------------------------------------------
	% MMD loss block (right side, with curly bracket)
	%-------------------------------------------------------
	\draw[-{Triangle[width=0.01pt,length=0.01pt]}, line width=2pt, color=\secondarycolor]
	(\xiv,-1.25) -- (\xiv, -2.3);
	\draw[-{Triangle[width=0.01pt,length=0.01pt]}, line width=2pt, color=\secondarycolor]
	(\xiv,-2.3) -- (\ximg - 0.2, -2.3);
	\draw[-{Triangle[width=4pt,length=6pt]}, line width=2pt, color=\secondarycolor]
	(\ximg - 0.2,-2.3) -- (\ximg - 0.2, -1);

	% Labels inside the MMD block
	\node at (\xiv, -0.7) {\hspace*{-2cm}$\left.\rule{0pt}{1.5cm}\right\}$};
	\node at (\xiv, -0.5) {MMD:};
	\node at (\xiv, -0.9) {$\mathcal{L}(\vec{x}, \vec{\theta})$};

	%-------------------------------------------------------
	% Left side: Random noise input
	%-------------------------------------------------------
	\node at (\xi, \yi)   {Random};
	\node at (\xi, \yii)  {noise};
	\node at (\xi, \yiii) {$\,\,\{\vec{x}_k\}$};

	%-------------------------------------------------------
	% Center: QAG circuit (imported small figure)
	%-------------------------------------------------------
	\node at (\ximg, \yimg) {
		\scalebox{.75}{\newcommand{\qau}{\gate[4, style={fill=\maincolor!30}]{\text{QAG}\big(\vec{\rule{0pt}{.24cm}x}, \vec{\theta}\big)}}
\begin{quantikz}[row sep={0.6cm,between origins}, wire types={q,q,n,q}]
	\qw & [-.2cm] \qau & [-.2cm] \qmeter{} \\
	\qw & \qw          & \qmeter{}         \\
	    &              & \mdots            \\
	\qw & \qw          & \qmeter{}
\end{quantikz}}
	};

	% Arrow from random noise → QAG
	\harr{-1.2}{0.1}

	%-------------------------------------------------------
	% Right side: Fake samples output
	%-------------------------------------------------------
	\node at (\xii, \yi)   {Fake};
	\node at (\xii, \yii)  {samples};
	\node at (\xii, \yiii) {$\,\,\{\tilde{A}_k\}$};

	% Arrow from QAG → Fake samples
	\harr{2.5}{0.1}

	%-------------------------------------------------------
	% Bottom left: True dataset
	%-------------------------------------------------------
	\node at (\xi,  \Yi)   {True};
	\node at (\xi,  \Yii)  {dataset};
	\node at (\xi,  \Yiii) {$\,\,\{E_k\}$};

	% Arrow from True dataset → Encoding
	\longharr{-1.4}{-1.6}

	% Label "Encoding" under the arrow
	\node at (\ximg - 1.2, -1.35) {\color{\tertiarycolor}\textbf{Encoding}};

	%-------------------------------------------------------
	% Bottom right: Encoded dataset
	%-------------------------------------------------------
	\node at (\xii,  \Yi)   {Encoded};
	\node at (\xii,  \Yii)  {dataset};
	\node at (\xii,  \Yiii) {$\,\,\{A_k\}$};

\end{tikzpicture}